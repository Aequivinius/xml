\documentclass[a4paper, 12pt]{scrartcl}
\usepackage{listings}
\usepackage{color}
\lstdefinelanguage{XML}
{
  basicstyle=\ttfamily\footnotesize,
  morestring=[b]",
  moredelim=[s][\bfseries\color{blue}]{<}{\ },
  moredelim=[s][\bfseries\color{blue}]{</}{>},
  moredelim=[l][\bfseries\color{blue}]{/>},
  moredelim=[l][\bfseries\color{blue}]{>},
  morecomment=[s]{<?}{?>},
  morecomment=[s]{<!--}{-->},
  commentstyle=\color{black},
  stringstyle=\color{blue},
  identifierstyle=\color{red}
}

\setkomafont{sectioning}{\rmfamily}

\title{XML Technologies: Exercise 1}
\date{}
\subtitle{Introduction to XML, DTDs}

\begin{document}
\maketitle\vspace{-12ex}

\noindent This is a mandatory exercise and the result will be part of your final mark. The solution must be uploaded to OLAT by \textbf{March 4\textsuperscript{th} at 15:59}. Late submissions will not be accepted.\\


\noindent Submit the following files in a zipped archive:
\begin{itemize}
\item wellformed.xml
\item validation.txt
\item singlemalt.dtd
\item valid.xml
\item invalid.xml
\end{itemize}

\noindent Name the archive [lastname]\_[firstname]\_1.zip (for example \textit{mueller\_mathias\_1.zip}). Some parts of your submission may be automatically evaluated, so make sure to name your files \textit{precisely} as prescribed, otherwise you might not get any points.

\section{Introducing XML and xmllint}

\subsection{xmllint (0 Points)}
\texttt{xmllint} is a program that parses XML files and, among other things, can perform \emph{validation}. It is installed by default on most UNIX-based operating systems, and by extension, on OSX, too. \\

\noindent If you are using one of these operationg systems, open your console and type:

\begin{lstlisting}[language=bash]
  xmllint --version
\end{lstlisting}

\noindent For Windows, there are some instructions\footnote{https://www.zlatkovic.com/libxml.en.html\#install} for how to install it. However, we cannot offer any support for these as part of this course.


% QUESTION 1: well-formed
\subsection{Well-Formedness (2 Points)}
Not all structured data is XML: The property of being ``correct'' XML is called well-formedness. The document \texttt{malformed.xml} is not well-formed. But what precisely is wrong? \\

\lstinputlisting[language=XML]{attachments/malformed.xml}

\noindent \textbf{Find all mistakes in the attempted XML document above and correct them. Save a well-formed version of it as \texttt{wellformed.xml}, explaining your corrections using \texttt{<!-- comments -->}. If all your objections are amended, the resulting document should indeed be well-formed XML.} \\

\noindent You may use \texttt{xmllint} to check your document for well-formedness. In order to do so, open your console, navigate to your directory and type:

\begin{lstlisting}[language=bash]
  xmllint -noout wellformed.xml
\end{lstlisting}

\noindent If \texttt{xmllint} doesn't complain, then the file is well-formed.

\section{Validation and DTD}

% QUESTION 2: questions about validation / well-formedness
\subsection{Validation and Valid Documents (1 Point)}

XML is not a markup language in its strict sense, but a \textbf{meta-markup language}. This is because XML does define a syntax to structure data, but does not define a vocabulary. A vocabulary is a collection of key words in the data that map to specific actions in applications. Likewise, an application might expect data items to be stored in a certain order, while XML allows the children of an element in any order. \\

\noindent \textbf{With all this in mind,
\begin{itemize}
\item What is validation and what purpose does it serve?
\item Under what conditions can an XML document be said to be valid?
\item Can XML documents be well-formed, but not valid?
\item Can XML documents be valid, but not well-formed?
\end{itemize}
Submit your answers in the text file \texttt{validation.txt}.}

\noindent 

% QUESTION 3: write a simple DTD
\subsection{Document Type Definitions (DTDs) (2 Points)}

Below is an example of a short, well-formed XML document. Remember, well-formedness has no say in whether the content makes sense.

\lstinputlisting[language=XML]{attachments/invalid.xml}


\noindent Single Malt must always be made from malted barley, and Bourbon is made from corn. \textbf{Write a DTD \texttt{singlemalt.dtd} where the following rules are laid down}:
\begin{itemize}
   \item A \texttt{single\_malt} element may only contain \texttt{whisky} or \texttt{whiskey} elements.
   \item Every \texttt{whisky} and \texttt{whiskey} element must have an \texttt{age} attribute.
   \item Every \texttt{whisky} and \texttt{whiskey} element may have an \texttt{origin} and \texttt{name} attribute.
\end{itemize}

Thus, while your DTD should \textit{invalidate} the XML above, the following document should be valid in regard to the DTD you have written:

\lstinputlisting[language=XML]{attachments/valid.xml}

\noindent Please may test your DTD by using \texttt{xmllint} as follows (or resort to some other method of validation):

\begin{lstlisting}[language=bash]
  xmllint --noout --dtdvalid singlemalt.dtd valid.xml
\end{lstlisting}

\noindent Also, to facilitate evaluation of your submission, also include the files \texttt{valid.xml} and \texttt{invalid.xml} in your submission.

\end{document}