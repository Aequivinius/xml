\documentclass[a4paper, 12pt]{scrartcl}
\usepackage{listings}
\usepackage{color}
\lstdefinelanguage{XML}
{
  basicstyle=\ttfamily\footnotesize,
  morestring=[b]",
  moredelim=[s][\bfseries\color{blue}]{<}{\ },
  moredelim=[s][\bfseries\color{blue}]{</}{>},
  moredelim=[l][\bfseries\color{blue}]{/>},
  moredelim=[l][\bfseries\color{blue}]{>},
  morecomment=[s]{<?}{?>},
  morecomment=[s]{<!--}{-->},
  commentstyle=\color{black},
  stringstyle=\color{blue},
  identifierstyle=\color{red}
}

\setkomafont{sectioning}{\rmfamily}


\title{XML Technologies: Exercise Solutions 1}
\date{}
\subtitle{Introduction to XML, DTDs}
 
\begin{document}
\maketitle\vspace{-5ex}

\noindent Questions and instructions are displayed in a font with normal weight, answers are set in \textbf{bold}. \\

\noindent Bear in mind that even though the sample solution might not consider your approach to solve a particular problem, this does by no means invalidate your solution. 

\section{Introducing XML}

% QUESTION 1: well-formedness of file (1 point)
\subsection{Well-Formedness (2 Points)}

Find all mistakes in the attempted XML document and correct them. \\

\lstinputlisting[language=XML]{attachments/wellformed.xml}

\section{Validation and DTD}

% QUESTION 2: questions about validation / well-formedness 
\subsection{Validation and Valid Documents (1 Points)}

\begin{itemize}
\item What is validation and what purpose does it serve? \\

\textbf{Validation is the process of checking a well-formed XML document against a set of rules that are usually defined either in a DTD, XML Schema or RelaxNG. The rules all relate to the structure, ordering or content of XML.}

\item Under what conditions can an XML document be said to be valid? \\

\textbf{An XML document is valid if it satisfies the rules defined in a DTD (or similar type of schema or definition language, for that matter).}

\item Can XML documents be well-formed, but not valid? \\

\textbf{Yes, an absolute majority of all possible XML documents will be invalid with respect to a particular DTD at all times.}

\item Can XML documents be valid, but not well-formed? \\

\textbf{No, an undefined something that is not well-formed XML cannot be validated.}
\end{itemize}

\noindent 

% QUESTION 3: write a simple DTD (1 point)
\subsection{Document Type Definitions (DTDs) (2 Points)}


Write a DTD where the following rules are laid down:
\begin{itemize}
   \item A \texttt{single\_malt} element may only contain \texttt{whisky} or \texttt{whiskey} elements.
   \item Every \texttt{whisky} and \texttt{whiskey} element must have an \texttt{age} attribute.
   \item Every \texttt{whisky} and \texttt{whiskey} element may have an \texttt{origin} and \texttt{name} attribute.
\end{itemize}


\lstinputlisting[language=XML]{attachments/singlemalt.dtd}



\end{document}